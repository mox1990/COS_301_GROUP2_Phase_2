% Margins
\topmargin=-0.45in
\evensidemargin=0in
\oddsidemargin=0in
\textwidth=6.5in
\textheight=9.0in
\headsep=0.25in

\linespread{1.1} % Line spacing

\subsection{Protocols}

\subsubsection{HTTP - HyperText Transfer Protocol}

The system will make use of HTTP to receive requests and send responses through the different access channels. This protocol will define the communication
between the sytem and web browsers as well the Android application.

\subsubsection{HTTPS - HyperText Transfer Protocol Secure}

The system will make use of HTTPS when communicating sensitive data through the different access channels.

\subsubsection{LDAP - Lightweight Directory Access Protocol}

The system will make use of the LDAP system used by the Department of Computer Science. This system contains the details of the users of the system.
Thus the users will be authenticated through this LDAP system. LDAP is both efficient and low cost. Making use of the LDAP protocol is an architectural constraint. 

\subsubsection{JSON - JavaScript Object Notation}

The system will make use of the JSON protocol to marshal response objects that are requested by other systems. JSON is the preferred
marshaling protocol due to it being efficient and lightweight. When compared to XML, the JSON marshaled objects are smaller in size
and the time taken to process them is less. These performace gains have made JSON the chosen marshaling protocol.

\subsubsection{REST - Representational State Transfer}

Strictly speaking REST is not a protocol, but it makes use of HTTP to provide web services. REST is preferred over SOAP as it is lightweight, easy to use and will provide
better performance in the case of this system. These advantages have made REST the chosen web services protocol. Making use of RESTful web services is an architectural
constraint.

\subsection{Libraries}